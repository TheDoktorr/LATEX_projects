    \documentclass[11pt]{article}
    \usepackage{amsmath}
    \usepackage{graphicx}
    \usepackage{multirow}
    \usepackage{booktabs}
    \usepackage{subfigure}
    \usepackage{verbatim}
    \usepackage{color}
    \usepackage{hyperref}
    \usepackage{url}



    \begin{document}

    \title{PHYS133 -- Lab report}
    \author{Andrew Hamill 39047415}
    \date{\today}
    \maketitle

    \begin{abstract}
    Radioactive sources decay randomly over time, with a fixed probability, and the half life describes the time take for a sample to halve its count rate. The count rates over a short period of time can also be modelled by statistical distributions, most commonly Poisson, in order to model the spread of decays, and calculate the Poisson standard deviation from this. The exact values of standard dev. and count rate for our source, U-238, were used to test the Poisson's fit, and determine the 2 values of standard deviation. Here it was shown a count rate $ R = 1.08 \pm 0.06$ Bq and $\sigma_{hist} = 1.71$. Our Poisson parameter was $m = 1.80$. This is very close to our standard deviation from the histogram, suggesting a good fit to the data, as well as the decay of U-238 being confirmed to be random, as expected. This is inline with the general accepted model of radioactive decays, which appear random over single time intervals, however can be modelled by a distribution when investigated over many repeated intervals. 
    \end{abstract}

    \section{Introduction}
    \section{background/Theory}
    \section{Experimental method/theoretical approach}
    \section{Results and analysis/discussion}
    \section{Conclusions}
    \section{References}
    \section{Appendices}
    
     \end{document}